\documentclass[11pt]{article}


\usepackage[utf8]{inputenc}
\usepackage{amssymb}
\usepackage{amsmath}
\usepackage{listings}

% \setlength\parindent{0pt}

\newcommand{\inlinecode}{\texttt}

\title{Languages and Paradigms of Programming \\
	\large Language specification
}
\author{Filip Binkiewicz (332069)}
\date{\today}
\begin{document}
\maketitle

\section{BNF Grammar}
\begin{align*}
	\textit{Num} \ni ~&n~::=~0~|~1~|~-1~|~2~|~\dots \\
	\textit{Var} \ni ~&x~::=~\texttt{x}~|~\texttt{y}~|~\dots \\
	\textit{Expr} \ni ~&e~::=~n~|~x~|~e_1 + e_2~|~e_1 * e_2~|~e_1 - e_2~|~e_1 / e_2~|\\
		& ~~~~~~~~~ f(x)~|~f(f_1)~|~x++~|~x-- \\
	\textit{BExpr} \ni ~&b~::=~\texttt{true}~|~\texttt{false}~|~x~|~e_1 <= e_2~|~e_1 < e_2~|\\
		& ~~~~~~~~~ e_1 >= e_2~|~e_1 > e_2~|~e_1 == e_2~|\\
		& ~~~~~~~~~ b_1~\texttt{and}~b_2~|~b_1~\texttt{or}~b_2~|~\texttt{not}~b\\
	\textit{FExpr} \ni ~&f~::=~x~|~\texttt{lambda (}x\texttt{):}~I\\
	\textit{Instr} \ni ~&I~::=~x = e~|~x = b~|~x = f~|\\
		& ~~~~~~~~~ \texttt{while}~b~\texttt{do}~I~|~I_1;I_2~|\\
		& ~~~~~~~~~ \texttt{if}~b~\texttt{then}~I~|~\texttt{if}~b~\texttt{then}~I_1~\texttt{else}~I_2~|\\
		& ~~~~~~~~~ \texttt{\{}~d~\texttt{;}~I~\texttt{\}}~|~\texttt{print}~e~|\\
		& ~~~~~~~~~ \texttt{for}~x~\texttt{ranging from}~e_1~\texttt{to}~e_2~\texttt{do}~I~|\\
		& ~~~~~~~~~ \texttt{return}~e~|~\texttt{yield}~e~\\
	\textit{Decl} \ni ~&d~::=~\texttt{int}~x~|~\texttt{bool}~x~|\\
		& ~~~~~~~~~ \texttt{function}~y\texttt{(}x\texttt{)}~I~|~\texttt{function}~y\texttt{(ref}~x\texttt{)}~I~|\\
		& ~~~~~~~~~ \texttt{function}~y\texttt{(fun}~x\texttt{)}~I~
\end{align*}

\section{Remarks}
\begin{itemize}
	\item Most of the constructions described in the grammar are standard and won't be explained further --- their meaning can be assumed to replicate their C's equivalents'. All non-trivial constructions are explained in the next section.
	\item This grammar has not yet been written in a format understood by \texttt{BNFC}, but it will be done shortly. The format imitates the one used in the Semantics and Verification class --- in particular, it does not take account for precedence level in expressions, and I omitted that on purpose, so as to make the grammar more human-readable. Rewriting it might imply some minor changes.
\end{itemize}
\section{Explanation}
Most of the features of this language have been taken from the standard list. Below I will describe the interesting features of this language.
\begin{itemize}
	\item General information.
	\begin{itemize}
		\item \textbf{One shall not use an undeclared variable!} (this is however not a syntax error as it is not a context-free feature)
		\item The language is \textbf{statically typed}, meaning exactly what the author of the assignment described so verbosely.
		\item Runtime errors are explicitly reported.
	\end{itemize}
	\item \texttt{if \_ then \_ else} - since there are two versions of this statement, the problem that bothered Pascal's developers appears. We assume that every \texttt{else} corresponds to the \textbf{closest} \texttt{if}, i.e. the program
	\begin{lstlisting}
	if false then
		if true then
			print 1
		else
			print 1
	\end{lstlisting}
does not print anything. But, of course, I endorse all programers not to take advantage of this too much.
	\item \texttt{function}. This language allows to declare three types of functions (all recursive with static visibility):
	\begin{itemize}
		\item pass-by-value
		\item pass-by-reference
		\item functions with a function parameter (pass-by-value)
	\end{itemize}
	All functions \textbf{return an integer value} (0 if a \texttt{return} statement is not met int the function's body).
	
	Another important remark is that a function declared as \texttt{function f(fun f)} discards the parameter and inside it every call of \texttt{f} will be a recursive call. Other than that, this is self-explainatory (I hope!).
	\item \texttt{lambda} denotes an anonymous function which can be assigned to a variable or passed to a function. Lambdas are only pass-by-value and return an integer.
	\item The Pascal-style \texttt{for} --- I believe this is self-explainatory as well.
	\item \texttt{print} allows to print an expression (an attempt to print a function variable will raise a ,,compile'' error.
	\item Blocks work like in C, except that there needs to be at least one semicolon - the one separating declarations from instructions. Hence, the following code is fine:
	\begin{lstlisting}
	{ ; print 10 }
	\end{lstlisting}
	\item \texttt{yield} works like in Python and is the only feature not described in the standard assignment specification. It allows the programmer to define rather stupid generators:
	\begin{lstlisting}
	function f(x)
	{
		int i;
		i = 0;
		while (true)
		{
			yield x + i;
			i = i + 1
		}
	}
	;
	print f(1);		// prints 1
	print f(1);		// prints 2
	print f(3)		// prints 5
		
	\end{lstlisting}
	A \texttt{yield} instruction outside of a function definition raises a runtime error.
	\item Comments - this will be done by \texttt{BNFC} and I think I'm going to stick with the \texttt{// comment} and \texttt{/* comment */} tradition.
\end{itemize}


\end{document}